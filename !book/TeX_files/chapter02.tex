\chapter{Matemática}

\section{Reglas de Divisibilidad}

Para determinar si un número es divisible por otro sin necesidad de hacer la división, se pueden usar las siguientes reglas de divisibilidad:

\subsection{Reglas de Divisibilidad Básicas}

\begin{enumerate}
	\item \textbf{Divisible por 1}: Todos los números son divisibles por 1.
	\item \textbf{Divisible por 2}: Un número es divisible por 2 si su último dígito es par (0, 2, 4, 6, 8).
	\item \textbf{Divisible por 3}: Un número es divisible por 3 si la suma de sus dígitos es divisible por 3.
	\item \textbf{Divisible por 4}: Un número es divisible por 4 si el número formado por sus últimos dos dígitos es divisible por 4.
	\item \textbf{Divisible por 5}: Un número es divisible por 5 si su último dígito es 0 o 5.
	\item \textbf{Divisible por 6}: Un número es divisible por 6 si es divisible por 2 y por 3.
	\item \textbf{Divisible por 7}: Hay varias reglas:
	\begin{itemize}
		\item \textbf{Regla 1}: Toma el último dígito, multiplícalo por 2 y réstalo del resto del número. Si el resultado es divisible por 7, entonces el número original también lo es.
		\item \textbf{Regla 2}: Divide el número en bloques de 3 dígitos (de derecha a izquierda) y calcula la suma alternante de estos bloques. Si el resultado es divisible por 7, entonces el número original también lo es.
	\end{itemize}
	\item \textbf{Divisible por 8}: Un número es divisible por 8 si el número formado por sus últimos tres dígitos es divisible por 8.
	\item \textbf{Divisible por 9}: Un número es divisible por 9 si la suma de sus dígitos es divisible por 9.
	\item \textbf{Divisible por 10}: Un número es divisible por 10 si su último dígito es 0.
\end{enumerate}

\subsection{Reglas de Divisibilidad Menos Comunes}

\begin{enumerate}
	\item \textbf{Divisible por 11}:
	\begin{itemize}
		\item \textbf{Regla 1}: Suma los dígitos en posiciones impares y resta la suma de los dígitos en posiciones pares. Si el resultado es divisible por 11, entonces el número original también lo es.
		\item \textbf{Regla 2}: Alternativamente, puedes usar la suma alternante de los dígitos.
	\end{itemize}
	\item \textbf{Divisible por 12}: Un número es divisible por 12 si es divisible por 3 y por 4.
	\item \textbf{Divisible por 13}:
	\begin{itemize}
		\item \textbf{Regla 1}: Toma el último dígito, multiplícalo por 9 y réstalo del resto del número. Si el resultado es divisible por 13, entonces el número original también lo es.
		\item \textbf{Regla 2}: Divide el número en bloques de 3 dígitos y calcula la suma alternante de estos bloques. Si el resultado es divisible por 13, entonces el número original también lo es.
	\end{itemize}
	\item \textbf{Divisible por 17}:
	\begin{itemize}
		\item \textbf{Regla 1}: Toma el último dígito, multiplícalo por 5 y réstalo del resto del número. Si el resultado es divisible por 17, entonces el número original también lo es.
	\end{itemize}
	\item \textbf{Divisible por 19}:
	\begin{itemize}
		\item \textbf{Regla 1}: Toma el último dígito, multiplícalo por 2 y súmalo al resto del número. Si el resultado es divisible por 19, entonces el número original también lo es.
	\end{itemize}
\end{enumerate}

\subsection{Reglas de Divisibilidad para Números Primos Mayores}

Para números primos mayores, como 23, 29, etc., las reglas pueden volverse más complejas y a menudo implican métodos similares a los usados para 7, 13, 17 y 19, pero con diferentes multiplicadores.

\subsection{Métodos Generales para Verificar Divisibilidad}

\begin{enumerate}
	\item \textbf{División Directa}: La forma más directa de verificar la divisibilidad es realizar la división y ver si el residuo es cero.
	\item \textbf{Uso de Factores}: Si un número es divisible por dos o más números coprimos, entonces es divisible por su producto. Por ejemplo, si un número es divisible por 3 y por 5, entonces es divisible por 15.
\end{enumerate}


\section{Números Primos}

\subsection{Chequear si un número es primo}

Complejidad: \( O(\sqrt{n}) \)
\begin{lstlisting}[language=C++, caption={Chequear si un número es primo}]
	bool esPrimo(int n) {
		if (n <= 1) return false;
		if (n == 2) return true;
		if (n % 2 == 0) return false;
		for (int i = 3; i * i <= n; i += 2)
		if (n % i == 0)
		return false;
		return true;
	}
\end{lstlisting}

