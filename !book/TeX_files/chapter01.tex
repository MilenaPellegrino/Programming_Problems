\chapter{Introducción}

\section{Sobre las notas}
Estas notas explican recordatorios y cosas basicas en programación competitiva. 
Fueron escritas con la idea de usarla como notebook en competencias. 

\section{Cosas de C++}

\subsection*{Lectura de líneas con espacios en C++}

En C++, para leer una línea completa que puede contener espacios (como una oración o una entrada con varios números separados), se utiliza:

\begin{verbatim}
	getline(std::cin, s);
\end{verbatim}

Sin embargo, si antes se usó \texttt{std::cin >>} (que deja un salto de línea \texttt{'\textbackslash n'} en el buffer), entonces \texttt{getline()} puede leer esa línea vacía. Para evitarlo, se debe limpiar el buffer con:

\begin{verbatim}
	cin.ignore();
\end{verbatim}


\subsection*{Tamaño de los tipos de datos}
\begin{itemize}
	\item \textbf{Tamaño de un long long}: Es un número de 64 bits, \( 2^{64} \) o aproximadamente \( 10^{19} \).
\end{itemize}
